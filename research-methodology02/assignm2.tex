\documentclass[a4paper]{article}
\usepackage[margin=0.5in]{geometry}
\usepackage{graphicx}
\usepackage{amsmath}
\usepackage{amsfonts}
\usepackage{amssymb}
\begin{document}

\begin{center}
\includegraphics[scale=0.2]{besm.jpg}\\

\includegraphics[scale=1.5]{mmulogo.png}\\ 
\end{center} 
\nopagebreak[3]

\begin{flushleft}
\begin{tabular}{|c| c| c|}
\hline
\large \bf Name & \large Mohamed Saleh & \large Loie Hesham \\
\hline
\large \bf Student ID & \large 1111113245 & \large 1091105774 \\
\hline
\end{tabular}  \newline \newline \newline 

{\Large Research Methodology Assignment-II \\
Tutorial Section TC209 \\
Number of Pagers is 3 (including this page) \\
Name of the Journal is {\bf Elsevier} {\it http://www.elsevier.com/journals} } \newline \newline \newline 
\end{flushleft} 

\tableofcontents


\title{Wireless Face Interface: Using voluntary gaze direction and facial muscle activations}
\author{ Outi Tuisku , Veikko Surakka , Toni Vanhala , Ville Rantanen , Jukka Lekkala}


\maketitle
\abstract{In this paper the authors will be presenting and testing a new human-computer interaction device, the device is called the wireless Face interface.The wireless face interface works as a pointing and selecting device using the gaze direction for pointing ,and facial muscles movement (Frowning or Rising the eyebrows)for selecting ,the device is developed based on several  other similar devises that have been developed earlier by different researchers.int the paper the authors will be testing the device and will be evaluating its performance with different evaluation methods.   } \\

\section{Problem Solved}
For many years developers and researchers have developed different devices and methods that detect the face and head motion ,and use them for human-
computer interaction purposes.Several methods were mentioned in the in this paper, each method had its own advantages and disadvantages, some of these 
disadvantages were the lake of accuracy, taking a longer period of time to execute a task that could be executed faster using other simpler methods 
(i.e.,pointing at an object on a computer screen using gaze and different face muscles detecting takes more time than using a computer mouse) , and 
complexity of the device's design   .Based on the earlier research and methods,and by learning from the earlier methods and researches weak points, the 
authors of this paper have put together a new device that detect voluntary gaze direction, and certain face muscles motion and employee them to point at 
and select objects on a computer screen with high accuracy and efficiency.

\section{Claimed Contributions}

The authors is presenting a new device called the wireless face interface device, the device is to detect the gaze direction and use it for pointing the 
objects on the screen,one of two facial techniques (Frowning or rising of the eye eyebrows) can be used as a selection signal. It consist of, a wearable 
protective glasses, the device is combination of a wearable video-based eye tracker and a capacitive sensor to detect the facial movement resulting either 
from the movement of the corrugator supercilii muscle which moves when frowning,and the frontalis muscle which moves when raising the eyebrows. the 
prototype contains two cameras one for one to capture the eye movement and the other one to image the computer screen ,an infrared (IR) light emitting 
diode for illumination of the eye and to provide the corneal reflection, a sensor for detecting facial movements using a capacitive method,and a shoulder 
bag which contained devices for wireless connection and transmitting,and a power supply unit. On the computer side, a separate receiving station is 
connected to the PC to receive all the captured data from the device. The advantages of the new device is a lot, one of them is that large majority could 
benefit from it.For example, people with disabilities should be able to use this technique, provided that normal eye movements and the ability to
move their facial muscles still remain. By comparing the Face Interface device to the earlier devices and studies, we find that its more accurate, faster, 
and reliable as the results of the experiments and assessments have reveled. As we mentioned earlier its a wireless device which will provide more freedom 
of movement to the user. It is also very easy to learn how to use the Face Interface device as it takes only a couple of minutes.
 

\section{Directly-related work} 
This paper has many directly-related work other than {\bf Surakka et al., 2004, 2005}, which this paper is an extension for it.
{\bf Chin et al. (2008) } was one of important related works that combained the use of gaze direction and facial EMG (electromyography)
in a different way than in {\bf Surakka et al., 2004, 2005} did. They used facial EMG to correct the inaccuracy of the eye tracker.They user select 
the object by gazing at the screen. If the cursor was not inside the object after the first step,then the user has to use facial movements while still 
gazing at the target to move the cursor. {\bf Fitts' law}, was used to compare different pointing methods with each other. Fitts' law gives us the 
difficulty of pointing task and its called the index of difficulty (ID) and can be calculated with $ID = log_2(\frac{A}{W} + 1 ) $  	  
and can be described as the movement time (MT) $ MT = a + b$ $ID$ where $a$ and $b$ are regression coefficients. An index of performance (IP)
value is calculated in $IP = \frac{1}{b} $ . {\bf Subjective rating} is as important as the Fitts' law, to collect the sublective rating of the 
used technique to see how the participants experience the used technique.


\section{Methodology}
	
There were 20 participants in experiments, from both genders. The range of ages was 19-43 years. All of the participants had a normal vision.
For the object selection, half of the participants preferred frowning and the other half by raising the eyebrows.The apparatus used in this experiment 
were, SamsungSyncMaster 24" for the screen with approximately 60 cm as viewing distance. Windows XP operating system was used to run the experiment.The 
experimental task of this study was an extension of the earlier studies.It starts with a home square and a target circle appeared at the same time on the 
screen. The participant task was to select the home square first, then the target circle.participant point on the first object(Home square) by gazing to it 
then select the object using there chosen selecting technique (frowning or raising the eyebrows) and repeat the same steps with the second object(target 
circle). The order of the selection and using the previous techniques is important so its has to be home square then target circle and not the opposite. 
each A pause of 2000 ms will happened after each successful selection, before the home square and the target circle appears again.The width of the home 
square was kept constant and it was 30 mm.The same process will be repeated but each time the location and distance of the object will change in order to 
measure the accuracy and speed of the prototype in the different positions and distances. First, the participants were introduced and then wore the 
equipment to check whether its ready for the experiment or not. Some practice had to be done before the actual experiment for about five minutes, on both 
selection techniques (frowning and raising the eyebrows).There was a short relaxation period before the actual experiment. The participant had to rate the 
experiment on the scales that were given. The scale varied from -4(e.g., bad experience) to +4(e.g., good experience).To analysis the data that have been 
collected using Mixed-model analyses of variance(ANOVA) were used.If an error occurs while clicking on the target circle, Bonferroni corrected {\it t}-
tests was used to detect that. The results of the pointing task time analyses using the post hoc pairwise comparisons showed that 40 mm diameter 
target had a faster pointing task time than for the 30 mm and 25 mm daimeter targets. The raising technique had a faster pointing task times than in the 
frowning technique.The results for the error rate analyses showed that the frownig technique had a higher mean error rate than the raising technique,
which makes the raising technique more efficient to use. Post hoc pairwise comparisons for the distance showed that the pointing distances of 450 and 520 
mm had more errors compared to the other pointing distances. Because the distances from 60 mm to 260 mm had the fastest and the most accurate mean pointing
task time, Fitts' law was performed on it.    	



\section{Conclusion and Future Works}
The Wireless Face Interface is built on frames of protective glasses it detect the voluntary gaze direction, and facial muscles movement when frowning or 
rising the eyebrows , the device was tested in a wider variety of tasks than the earlier methods and studies have done .The results of testing the two 
different techniques of selecting (Frowning and rising the eyebrows) showed that both of them work well, resulting from that the users will have the 
freedom to chose which technique they prefer. Based on the results the device performs well in wide ring on the screen but it performs even better on a 
smaller range. The device show's potentials to be a great replacement for the regular mouse or any other traditional pointing and selecting methods in the 
future , as it has several good qualities that qualifies it to be a successful and popular HCI device.  


\section{What we have learned?}
We have learned many things from this paper. One of the things that we have learned is ANOVA, which is Mixed-model analyses of variance and its used
as a statistical models to collect and analyze the differences between group of means and their associated procedures. 



\section{Reference}
\begin{enumerate}
\item Barreto, A.B., Scargle, S.D., Adjouadi, M., 2000. A practical EMG-based human–computer interface for users with motor disabilities. J. Rehabil. Res. 
Dev. 37, 53–63.
\item Bradley, M., Lang, P.J., 1994. Measuring emotion: the self-assessment manikin and the semantic differential. J. Behav. Ther. Exp. Psychiatr. 25, 49–
59.
\item Bradski, G., Kaehler, A., 2008. Learning OpenCV: Computer Vision with the OpenCV Library. O’Reilly Media, Sebastopol, California, USA.

\end{enumerate}


\end{document}

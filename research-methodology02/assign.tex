\documentclass[a4paper]{article}
\usepackage[margin=0.5in]{geometry}
\usepackage{graphicx}
\usepackage{amsmath}
\usepackage{amsfonts}
\usepackage{amssymb}
\begin{document}

\begin{center}
\includegraphics[scale=0.2]{besm.jpg}\\

\includegraphics[scale=1.5]{mmulogo.png}\\ 
\end{center} 
\nopagebreak[3]

\begin{flushleft}
\begin{tabular}{|c| c| c|}
\hline
\large \bf Name & \large Mohamed Saleh & \large Loie Hesham \\
\hline
\large \bf Student ID & \large 1111113245 & \large 1091105774 \\
\hline
\end{tabular}  \newline \newline \newline 

{\Large Research Methodology Assignment-I \\
Tutorial Section TC209 \\
Number of Pagers is 3 (including this page) \\
Name of the Journal is {\bf Elsevier} {\it http://www.elsevier.com/journals} } \newline \newline \newline 
\end{flushleft} 

\tableofcontents


\title{Wireless Face Interface: Using voluntary gaze direction and facial muscle activations}
\author{ Outi Tuisku , Veikko Surakka , Toni Vanhala , Ville Rantanen , Jukka Lekkala}


\maketitle
\abstract{ the gvoajg  } \\

\section{Problem Solved}
For many years developers and researchers have developed different devices and methods that detect the face and head motion ,and use them for human-computer interaction purposes.Several methods were mentioned in the in this paper ,each method had its own advantages and disadvantages, some of these disadvantages were the lake of accuracy , taking a longer period of time to execute a task that could be executed faster using other simpler methods (i.e:pointing at an object on a computer screen using gaze and different face muscles detecting takes more time than using a computer mouse) , and complexity of the device's design   .Based on the earlier research and methods,and by learning from the earlier methods and researches weak points, the authors of this paper have put together a new device that detect voluntary gaze direction, and certain face muscles motion and employee them to point at and select objects on a computer screen with high accuracy and efficiency.

\section{Claimed Contributions}
the main objective was creating a device 
 


\section{Directly-related work} 

\section{Methodology}
	
There were 20 participants in experiments, from both genders. The range of ages was 19-43 years. All of the participants had a normal vision.
For the object selection, half of the participants preferred frowning and the other half by raising the eyebrows.The apparatus used in this experiment were, SamsungSyncMaster 24" for the screen with approximately 60 cm as viewing distance. Windows XP operating system was used to run the experiment.The experimental task of this study was an extension of the earlier studies.It starts with a home square and a target circle appeared at the same time on the screen. The participant task was to select the home square first, then the target circle.participant point on the first object(Home square) by gazing to it then select the object using there chosen selecting technique (frowning or raising the eyebrows) and repeat the same steps with the second object(target circle). The order of the selection and using the previous techniques is important so its has to be home square then target circle and not the opposite. each A pause of 2000 ms will happened after each successful selection, before the home square and the target circle appears again.The width of the home square was kept constant and it was 30 cm.The same process will be repeated but each time the location and distance of the object will change in order to measure the accuracy and speed of the prototype in the different positions and distances.      
 
\section{Conclusion}



\section{Reference}


\end{document}
